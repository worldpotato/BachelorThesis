\chapter{Introduction}\label{ch:introduction}

\section{Motivation}\label{sec:motivation}

Mobile mapping solutions often use multiple sensors and combine the measurements of these sensors to a map and a position inside this map.
Common sensors are laser scanner, stereo camera and \ac{IMU}.
But also \ac{GNSS} receiver are often used for outdoor applications.
From two-dimensional distance measurements from laser scanner to the orientation, velocity and acceleration from an \ac{IMU}, these sensors provide a wide range of different data.
To enhance the amount of data types and methods a thermal camera can be added to the mobile mapping system.
This additional sensor gives possibilities to use further methods of photogrammetry.

Therefore a lot of different methods are used to achieve the goal of creating a map and locate the own position in this map.
The amount of different methods offer many possibilities for tutors and professors to give their students a hands-on experience of what they learned in the last lessons.
But the sensor systems are also used in student projects, bachelor and master theses.
Mostly with different approaches and different outcomes.

It takes a lot of time to understand the implementation of different projects.
And the different approaches lead to more complexity while setting up new projects or adapting an old project for new lectures which raise the possibly to make errors.
Another side effect of different outcomes is the difficulty to combine two projects to one bigger project.

\section{Related Work}\label{sec:related-work}
Developing Intelligent Environments: A Development Tool Chain for Creation, Testing and Simulation of Smart and Intelligent Environments\cite{roalter2011developing}

Robotik-lokalization\cite{MooreStouchKeneralizedEkf2014}

ROS:\@ an open source robot operating system\cite{Quigley2009ROSAO}

\section{Aim of the Work}\label{sec:aimOfTheWork}
\section{Structure}\label{sec:structure}

