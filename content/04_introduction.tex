\chapter{Introduction}\label{ch:introduction}

\section{Motivation}\label{sec:motivation}

Mobile mapping solutions often use multiple sensors and combine the measurements of these sensors to a map and position.
Common sensors for mobile mapping solutions are laser scanner, stereo camera and \ac{IMU}.
But also \ac{GNSS} receiver are often used for outdoor applications.
The sensors provide a wide range of different data.
And using the provided data for positioning and map building requires different methods.

In universities different methods to process sensor data are taught during different lectures from different professors and tutors.
And many of them use mobile mapping solutions to give their students a hands-on experience to these methods.
Others add sensors to customize the multi sensor network for their needs.
One mentionable example for a additional sensor is a \ac{IR} camera to use further methods of photogrammetry in a mobile mapping solution.
But mobile mapping platforms are not only used for teaching methods to students but also for research in different geodesy disciplines.
Some of the recent researches are mentioned in section related work[\ref{sec:related-work}].

Sometimes different scientists try to merge their research to one bigger project.
The process of merging two different works is often very complex because the two implementation differs a lot.
But not only the implementation of similar methods can differ from project to project, also the defined input and output format can be different.
As different two projects are as bigger the complexity of merging gets.
And the higher the complexity is, the higher is the chance to make failures what can cost a lot of time for debugging.

Working together on one project always means to define well known interfaces between the different parts.
Only setting up the interfaces can take a lot of time.

\section{Related Work}\label{sec:related-work}
Developing Intelligent Environments: A Development Tool Chain for Creation, Testing and Simulation of Smart and Intelligent Environments\cite{roalter2011developing}

Robotik-lokalization\cite{MooreStouchKeneralizedEkf2014}

ROS:\@ an open source robot operating system\cite{Quigley2009ROSAO}

\section{Aim of the Work}\label{sec:aimOfTheWork}
\section{Structure}\label{sec:structure}

