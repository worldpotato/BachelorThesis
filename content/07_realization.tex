\chapter{Realization}\label{ch:realization}

\section{Data Acquisition}\label{ch:realization:sec:dataAcquisition}
	Wie werden die Daten der Sensoren erfasst und vom software environment verarbeitet.
	Dabei immer darauf eingehen warum ROS\@.
	\subsection{Data Exchange}\label{ch:realization:ssec:dataExchange}
	Wie kommen Daten von den Sensoren (publisher) zu den weiter verarbeitenden Nodes (subscriber). Warum ist das gut so.
	\subsection{Storing Data}\label{ch:realization:ssec:storingData}
	Hier möchte ich vor allem auf die Bag file eingehen.
	\subsection{Visualization}\label{ch:realization:ssec:visualization}
	Die Visualisierung ist recht interessant, da sie einen ersten anhaltspunkt gibt wie die Daten aussehen. Außerdem ist es für den Nutzen im Lehrbetrieb der Hochschule gut den Studenten zu verdeutlichen was die Sensoren für Daten erzeugen und wie diese Aussehen.
	Zudem passiert dafür noch einige Transformationen und man kann hier noch ein paar Vorteile von Ros erwähnen.
\section{Data Processing}\label{ch:realization:sec:dataProcessing}
	Wie werden die Daten nach dem Abspeichern als Bag file genutzt Matlab part
	Außerdem warum Matlab genutzt wurde
	\subsection{Online}\label{ch:realization:ssec:online}
	\subsection{Offline}\label{ch:realization:ssec:offline}
